% resumo em inglês
\setlength{\absparsep}{18pt} % ajusta o espaçamento dos parágrafos do resumo
\begin{resumo}[Abstract]
 \begin{otherlanguage*}{english}
   Knowing that the behavior of relative humidity, intergranular temperature, and CO2 concentration inside the silo bag is a determining factor for the final quality of grains, having the ability to monitor these data is of utmost importance to act at the right moment and avoid significant losses. Considering the need to use silo bags to alleviate the deficit in static grain storage capacity in the country, this project aimed to develop a low-cost prototype to measure intergranular data of corn grains stored in silo bags, monitoring temperature, relative humidity, and CO2 concentration in the intergranular air. The prototype was developed through 3D modeling and printing, using embedded sensors to capture the data. An adaptation of the silo bag was necessary, but the results obtained were as expected, successfully identifying the temperature variation of the grain mass, ranging from 24.8°C to 27°C, and obtaining the intergranular air relative humidity values that remained practically constant at 68\% RH, compared to the external humidity, which varied between 57\% and 80\% RH. It was not possible to use the CCS811 carbon dioxide sensor due to its inability to measure the high CO2 concentrations inside the silo bag. The developed prototype met expectations, delivering good results and being a low-cost equipment, costing only R\$ 253.90.
   

   \vspace{\onelineskip}
 
   \noindent 
   \textbf{Keywords}: silo bag, grains, monitoring.
 \end{otherlanguage*}
\end{resumo}
% resumo em português
\setlength{\absparsep}{18pt} % ajusta o espaçamento dos parágrafos do resumo
\begin{resumo}
 
Sabendo que o comportamento da umidade relativa, temperatura intergranular e concentração de CO2 no interior do silo bolsa é um fator determinante para a qualidade final dos grãos, ter como monitorar esses dados é de extrema importância para poder agir no momento correto e evitar perdas significativas. E visando a necessidade do uso de silo bolsa para amenizar o déficit de capacidade estática de armazenagem de grãos no país, o presente projeto teve como objetivo desenvolver um protótipo de baixo custo para medir dados intergranular de grãos de milho armazenados em silo bolsas, monitorando a temperatura, a umidade relativa e concentração de CO2 do ar intergranular. O protótipo foi desenvolvido através de modelagem e impressão 3D, usando sensores embarcados para captar os dados. Foi necessário fazer uma adaptação do silo bolsa, porém os resultados obtidos foram dentro do esperado, conseguindo identificar a variação de temperatura da massa de grãos que variou de 24,8°C à 27°C e obter os valores de umidade relativa do ar intergranular que permaneceu praticamente constante em 68\% UR em comparação com a externa que variou entre 57\% e 80\% UR. Não foi possível usar o sensor CCS811 de dióxido de carbono devido ao sensor não conseguir mediar as altas concentrações de CO2 no interior do silo bolsa. O protótipo desenvolvido atendeu a expectativa apresentando bons resultados e sendo um equipamento de baixo custo custando apenas R\$ 253,90.

 \textbf{Palavras-chave}: silo bolsa, grãos, monitoramento.
\end{resumo}